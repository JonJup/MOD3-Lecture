\documentclass{article}

\setlength{\parindent}{0cm}

\begin{document}
This documents contains literature that was either cited in the MOD3 Lecutre or that I think is helpful to deepen your understanding of the discussed methods. The refernces are arranged in thematic groups following the same structure as the lecture. References within each group are order alphabetically. As some references are pertinent to multiple topics you will find repetitions. 
\section{Comparisions of model and distance based approaches}
Carlos-Júnior, L. A., Creed, J. C., Lewis, R. J., Marrs, R. H., Moulton, T. P., Feijó-Lima, R., \& Spencer, M. (2020). Generalized Linear Models outperform commonly used canonical analysis in estimating spatial structure of presence/absence data. Peer J, 8 \\

Hui, F. K. C., Taskinen, S., Pledger, S., Foster, S. D., \& Warton, D. I. (2015). Model-based approaches to unconstrained ordination. Methods in Ecology and Evolution, 6(4), 399–411. \\

Jupke, J. F., \& Schäfer, R. B. (2020). Should ecologists prefer model‐ over distance‐based multivariate methods? Ecology and Evolution\\

Roberts, D. W. (2017). Distance, dissimilarity, and mean-variance ratios in ordination. Methods in Ecology and Evolution. \\

Roberts, D. W. (2019). Comparison of distance-based and model-based ordinations. Journal of Ecology. \\

Szöcs, E., Van den Brink, P. J., Lagadic, L., Caquet, T., Roucaute, M., Auber, A., et al. (2015). Analysing chemical-induced changes in macroinvertebrate communities in aquatic mesocosm experiments: a comparison of methods. Ecotoxicology, 24(4), 760–769.\\ 

Warton, D. I., Wright, S. T., \& Wang, Y. A. (2012). Distance-based multivariate analyses confound location and dispersion effects. Methods in Ecology and Evolution, 3(1), 89–101.  \\

Warton, D. I., Foster, S. D., De’ath, G., Stoklosa, J., \& Dunstan, P. K. (2015). Model-based thinking for community ecology. Plant Ecology, 216(5), 669–682. \\

Warton, D. I., \& Hui, F. K. C. (2017). The central role of mean-variance relationships in the analysis of multivariate abundance data: a response to Roberts (2017). Methods in Ecology and Evolution, 8(11), 1408–1414. \\

Yamaura, Y., Blanchet, F. G., \& Higa, M. (2019). Analyzing community structure subject to incomplete sampling: hierarchical community model vs canonical ordinations. Ecology\\
\section{mvabund}
Dunn, P. K., \& Smyth, G. K. (1996). Randomized Quantile Residuals. Journal of Computational and Graphical Statistics, 5(3), 236–244. \\

Wang, Y. A. et al. (2012). Mvabund- an R package for model-based analysis of multivariate abundance data. Methods in Ecology and Evolution, 3(3), 471–474. \\

Warton D.I. (2008). Penalized normal likelihood and ridge regularization of correlation and covariance matrices. Journal ofthe American Statistical Association 103, 340-349.\\ 

Warton D.I. (2011). Regularized sandwich estimators for analysis of high dimensional data using generalized estimating equations. Biometrics, 67(1), 116-123.\\
\section{VGLM}
Baselga, A., \& Araújo, M. B. (2009). Individualistic vs community modelling of species distributions under climate change. Ecography, 32(1), 55–65. \\

Bonthoux, S., Baselga, A., \& Balent, G. (2013). Assessing Community-Level and Single-Species Models Predictions of Species Distributions and Assemblage Composition after 25 Years of Land Cover Change. \textbf{\textit{PLoS ONE, 8(1)}}. \\

Dobson, A. J., \& Barnett, A. G. (2018). An introduction to generalized linear models. CRC press.\\

Maguire, K. C. \textit{et. al} (2016). Controlled comparison of species- and community-level models across novel climates and communities. \textbf{\textit{Philosophical Transactions of the Royal Society B}}, 283, 20152817\\

Oksanen, J., \& Minchin, P. R. (2002). Continuum theory revisited: What shape are species responses along ecological gradients? Ecological Modelling, 157(2–3), 119–129.\\

ter Braak, C. J. F. 1986. Canonical correspondence analysis: A new eigenvector technique for multivariate direct gradient analysis. Ecology 67(5):1167–1179.\\ 

ter Braak, C. J. F., \& Šmilauer, P. (2014). Topics in constrained and unconstrained ordination. Plant Ecology, 216(5), 683–696.\\ 

Top, N., Tarkan, A. S., Vilizzi, L., \& Karakuş, U. (2016). Microhabitat interactions of non-native pumpkinseed Lepomis gibbosus in a Mediterranean-type stream suggest no evidence for impact on endemic fishes. Knowledge \& Management of Aquatic Ecosystems, 417(36), 1–7.\\


Vilizzi, L., Stakenas, S., \& Copp, G. H. (2012). Use of constrained additive and quadratic ordination in fish habitat studies: an application to introduced pumpkinseed Lepomis gibbosus and native brown trout Salmo truttain an English stream. Fundamental and Applied Limnology, 180(1), 69–75. \\

Yee, T. W. (2002). Vector generalized additive models in plant ecology. Ecological Modelling, 157(2–3), 141–156.\\

Yee, T. W. (2004). A New Technique for Maximum-Likelihood Canonical Gaussian Ordination. Ecological Monographs, 74(4), 685–701.\\

Yee, T. W. (2006). Constrained additive ordination. Ecology, 87(1), 203–213. \\

Yee, T. W. (2010). VGLMs and VGAMs: An overview for applications in fisheries research. Fisheries Research, 101(1–2), 116–126. \\

Yee, T. W. (2015). Vector generalized linear and additive models: with an implementation in R. Springer.\\

\section{Latent Variable Models}

Hoegh, A. \& Roberts, D. W. (2020). Evaluating and presenting uncertainty in model-based unconstrained ordination. Ecology and Evolution, 10(1), 59–69.\\

Hui, F. K. C. \textit{et al.} (2016). Variational Approximations for Generalized Linear Latent Variable Models. Journal of Computational and Graphical Statistics, 26(1), 35–43.\\ 

Hui, F. K. C. (2017). Model-based simultaneous clustering and ordination of multivariate abundance data in ecology. Computational Statistics and Data Analysis, 105, 1–10. \\

Letten, A. D. \textit{et al.} (2015). Fine-scale hydrological niche differentiation through the lens of multi-species co-occurrence models. Journal of Ecology, 103(5). \\ 

Niku, J. \textit{et al.} (2017). Generalized Linear Latent Variable Models for Multivariate Count and Biomass Data in Ecology. Journal of Agricultural, Biological, and Environmental Statistics, 1–25. \\

Niku, J. \textit{et al.} (2019). gllvm – Fast analysis of multivariate abundance data with generalized linear latent variable models in R. Methods in Ecology and Evolution, 1(1), 1–2. \\

Niku, J. (2019). On modeling multivariate abundance data with generalized linear latent variable models.

Ovaskainen, O. \textit{et al.} (2016). Using latent variable models to identify large networks of species-to-species associations at different spatial scales. Methods in Ecology and Evolution, 7(5), 549–555. \\

Warton, D. I. \textit{et al.} (2015). So Many Variables: Joint Modeling in Community Ecology. Trends in Ecology & Evolution, 30(12), 766–779. 

\section{copula}

Anderson, M. J. \textit{et al.} (2019). A pathway for multivariate analysis of ecological communities using copulas. Ecology and Evolution, (September 2018), 1–19.\\

Ghosh, S. \textit{et al.} (2020). Copulas and their potential for ecology. Advances in Ecological Research, 62, 409–468. \\

Ghosh, S. \textit{et al.} (2020). Tail associations in ecological variables and their impact on extinction risk. Ecosphere 11(5):e03132. \\

Popovic, G. C. \textit{et al.} (2019). Untangling direct species associations from indirect mediator species effects with graphical models. Methods in Ecology and Evolution, 2019(June), 1571–1583. \\

Sklar, A. (1959). Fonctions de répartition à n dimensions et leurs marges. Publications De L’institut De Statistique De L’université De Paris, 8, 229–231\\ 

\section{HMSC}

Norberg, A. \textit{et al.} (2019). A comprehensive evaluation of predictive performance of 33 species distribution models at species and community levels. Ecological Monographs, 89(3). \\

\end{document}