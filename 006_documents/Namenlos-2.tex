\documentclass{article}



\begin{document}
This documents contains literature that was either cited in the MOD3 Lecutre or that I think is helpful to deepen your understanding of the discussed methods. The refernces are arranged in thematic groups following the same structure as the lecture. References within each group are order alphabetically. As some references are pertinent to multiple topics you will find repetitions. 
\section{Comparisions of model and distance based approaches}
Carlos-Júnior, L. A., Creed, J. C., Lewis, R. J., Marrs, R. H., Moulton, T. P., Feijó-Lima, R., \& Spencer, M. (2020). Generalized Linear Models outperform commonly used canonical analysis in estimating spatial structure of presence/absence data. Peer J, 8 \\

Hui, F. K. C., Taskinen, S., Pledger, S., Foster, S. D., \& Warton, D. I. (2015). Model-based approaches to unconstrained ordination. Methods in Ecology and Evolution, 6(4), 399–411. \\
Jupke, J. F., \& Schäfer, R. B. (2020). Should ecologists prefer model‐ over distance‐based multivariate methods? Ecology and Evolution\\

Roberts, D. W. (2017). Distance, dissimilarity, and mean-variance ratios in ordination. Methods in Ecology and Evolution. \\

Roberts, D. W. (2019). Comparison of distance-based and model-based ordinations. Journal of Ecology. \\

Szöcs, E., Van den Brink, P. J., Lagadic, L., Caquet, T., Roucaute, M., Auber, A., et al. (2015). Analysing chemical-induced changes in macroinvertebrate communities in aquatic mesocosm experiments: a comparison of methods. Ecotoxicology, 24(4), 760–769.\\ 

Warton, D. I., Wright, S. T., \& Wang, Y. A. (2012). Distance-based multivariate analyses confound location and dispersion effects. Methods in Ecology and Evolution, 3(1), 89–101.  \\

Warton, D. I., Foster, S. D., De’ath, G., Stoklosa, J., \& Dunstan, P. K. (2015). Model-based thinking for community ecology. Plant Ecology, 216(5), 669–682. \\

Warton, D. I., \& Hui, F. K. C. (2017). The central role of mean-variance relationships in the analysis of multivariate abundance data: a response to Roberts (2017). Methods in Ecology and Evolution, 8(11), 1408–1414. \\

Yamaura, Y., Blanchet, F. G., \& Higa, M. (2019). Analyzing community structure subject to incomplete sampling: hierarchical community model vs canonical ordinations. Ecology\\
\section{mvabund}
Wang, Y. A. et al. (2012). Mvabund- an R package for model-based analysis of multivariate abundance data. Methods in Ecology and Evolution, 3(3), 471–474. \\

\end{document}