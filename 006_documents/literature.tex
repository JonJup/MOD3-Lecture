\documentclass{article}

\setlength{\parindent}{0cm}

\title{References for MOD3: Advances in Multivariate Statistics}

\begin{document}

\maketitle

This documents contains literature that was either cited in the MOD3 Lecutre or that I think is helpful to deepen your understanding of the discussed methods. The refernces are arranged in thematic groups following the same structure as the lecture. References within each group are ordered alphabetically. As some references are pertinent to multiple topics you will find repetitions. In the very end there is an additional section pointing you towards methods that were not dicussed but are noteworthy and interesting nontheless.   

\section{What’s new in multivariate analysis?}

Brittain, J. E. \textit{et al.} (2020). Ecological correlates of riverine diatom and macroinvertebrate alpha and beta diversity across Arctic Fennoscandia. Freshwater Biology, (January 2019), 1–15\\

Carlos-Júnior, L. A. \textit{et al.} (2020). Generalized Linear Models outperform commonly used canonical analysis in estimating spatial structure of presence/absence data. Peer J, 8 \\

Hui, F. K. C. \textit{et al.} (2015). Model-based approaches to unconstrained ordination. Methods in Ecology and Evolution, 6(4), 399–411. \\

Jupke, J. F., \& Schäfer, R. B. (2020). Should ecologists prefer model‐ over distance‐based multivariate methods? Ecology and Evolution\\

Paul, W. (2020). Ecological Informatics Covariate-adjusted species response curves derived from long-term macroinvertebrate monitoring data using classical and contemporary model-based ordination methods. Ecological Informatics, 60, 101159.\\

Roberts, D. W. (2017). Distance, dissimilarity, and mean-variance ratios in ordination. Methods in Ecology and Evolution. \\

Roberts, D. W. (2019). Comparison of distance-based and model-based ordinations. Journal of Ecology. \\

Szöcs, E. \textit{et al.} (2015). Analysing chemical-induced changes in macroinvertebrate communities in aquatic mesocosm experiments: a comparison of methods. Ecotoxicology, 24(4), 760–769.\\ 

Warton, D. I. \textit{et al.} (2012). Distance-based multivariate analyses confound location and dispersion effects. Methods in Ecology and Evolution, 3(1), 89–101.  \\

Warton, D. I. \textit{et al.} (2015). Model-based thinking for community ecology. Plant Ecology, 216(5), 669–682. \\

Warton, D. I., \& Hui, F. K. C. (2017). The central role of mean-variance relationships in the analysis of multivariate abundance data: a response to Roberts (2017). Methods in Ecology and Evolution, 8(11), 1408–1414. \\

Yamaura, Y. \textit{et al.} (2019). Analyzing community structure subject to incomplete sampling: hierarchical community model vs canonical ordinations. Ecology\\

Yee, T. W. (2004). A New Technique for Maximum-Likelihood Canonical Gaussian Ordination. Ecological Monographs, 74(4), 685–701.\\


\section{mvabund}

Bates D. \textit{et al.} (2015). Fitting Linear Mixed-Effects Models Using lme4. Journal of Statistical Software, 67(1), 1-48. \\

Dunn, P. K., \& Smyth, G. K. (1996). Randomized Quantile Residuals. Journal of Computational and Graphical Statistics, 5(3), 236–244. \\

Dunn, P. K., \& Smyth, G. K. (2018). Generalized linear models with examples in R. New York, NY: Springer.\ \ 

Hadfield, J. D. (2010). MCMC Methods for Multi-Response Generalized Linear Mixed Models: The MCMCglmm R Package. Journal of Statistical Software, 33(2), 1-22. \\

Pollock, L. J. \textit{et al.} (2014). Understanding co-occurrence by modelling species simultaneously with a Joint Species Distribution Model (JSDM). Methods in Ecology and Evolution, 5(5), 397–406.\\

Wang, Y. A. \textit{et al.} (2012). Mvabund- an R package for model-based analysis of multivariate abundance data. Methods in Ecology and Evolution, 3(3), 471–474. \\

Warton, D.I. (2008a). Penalized normal likelihood and ridge regularization of correlation and covariance matrices. Journal ofthe American Statistical Association 103, 340-349.\\ 

Warton, D. I. (2008b). Raw data graphing: An informative but under-utilized tool for the analysis of multivariate abundances. Austral Ecology, 33(3), 290–300.  \\

Warton D.I. (2011). Regularized sandwich estimators for analysis of high dimensional data using generalized estimating equations. Biometrics, 67(1), 116-123.\\

\section{Constrained Quadratic Ordination}

Baselga, A. \& Araújo, M. B. (2009). Individualistic vs community modelling of species distributions under climate change. Ecography, 32(1), 55–65. \\

Bonthoux, S. \textit{et al.} (2013). Assessing Community-Level and Single-Species Models Predictions of Species Distributions and Assemblage Composition after 25 Years of Land Cover Change. \textit{PLoS ONE, 8(1)}. \\

Dobson, A. J. \& Barnett, A. G. (2018). An introduction to generalized linear models. CRC press.\\

Everitt, B. S. \& Skrondal. A. (2010). The Cambridge Dictionary of Statistics. Cambridge University Press.\\

Gauch Jr., H. G. \& Whittaker, R. H. (1972). Comparison of ordination techniques. Ecology, 53(5), 868–875. \\

Holyoak M. \& Wetzel W. C. (2020) Variance Explict Ecology. In: Dobson, A., Tilman, D., \& Holt, R. D. (2020). Unsolved Problems in Ecology. Princeton University Press.\\

Maguire, K. C. \textit{et. al} (2016). Controlled comparison of species- and community-level models across novel climates and communities. \textit{Philosophical Transactions of the Royal Society B}, 283, 20152817\\

Oksanen, J. \& Minchin, P. R. (2002). Continuum theory revisited: What shape are species responses along ecological gradients? Ecological Modelling, 157(2–3), 119–129.\\

Ovaskainen, O., \textit{et al.} (2017b). How are species interactions structured in species-rich communities? A new method for analysing time-series data. Proceedings of the Royal Society B: Biological Sciences, 284(1855).\\

ter Braak, C. J. F. 1986. Canonical correspondence analysis: A new eigenvector technique for multivariate direct gradient analysis. Ecology 67(5):1167–1179.\\ 

ter Braak, C. J. F. \& Šmilauer, P. (2014). Topics in constrained and unconstrained ordination. Plant Ecology, 216(5), 683–696.\\ 

Top, N. \textit{et al.} (2016). Microhabitat interactions of non-native pumpkinseed Lepomis gibbosus in a Mediterranean-type stream suggest no evidence for impact on endemic fishes. Knowledge \& Management of Aquatic Ecosystems, 417(36), 1–7.\\

Vilizzi, L. \textit{et al.} (2012). Use of constrained additive and quadratic ordination in fish habitat studies: an application to introduced pumpkinseed Lepomis gibbosus and native brown trout Salmo truttain an English stream. Fundamental and Applied Limnology, 180(1), 69–75. \\

Yee, T. W. \& Wild, C. J. (1996). Vector Generalized Additive Models. Journal of the Royal Statistical Society. Series B, 58(3), 481–493.\\

Yee, T. W. (2002). Vector generalized additive models in plant ecology. Ecological Modelling, 157(2–3), 141–156.\\

Yee, T. W. (2004). A New Technique for Maximum-Likelihood Canonical Gaussian Ordination. Ecological Monographs, 74(4), 685–701.\\

Yee, T. W. (2006). Constrained additive ordination. Ecology, 87(1), 203–213. \\

Yee, T. W. (2010). VGLMs and VGAMs: An overview for applications in fisheries research. Fisheries Research, 101(1–2), 116–126. \\

Yee, T. W. (2015). Vector generalized linear and additive models: with an implementation in R. Springer.\\

Zuur, A. \textit{et al.} (2007). Analyzing ecological data. Springer.\\

\section{Latent Variable Models}

Blanchet, F. G. \textit{et al.} (2020). Co-occurrence is not evidence of ecological interactions. Ecology Letters, 23(7), 1050–1063.

Damgaard, C. \textit{et al.} (2020). Model-based ordination of pin-point cover data : effect of management on dry heathland. BioRxiv, 1–19.\\

Dormann, C. F. \textit{et al.} (2018). Biotic interactions in species distribution modelling: ten questions to guide interpretation and avoid false conclusions. Global Ecology and Biogeography, 1–13.\\

Han, Z. \textit{et al.} (2020). Unravelling species co-occurrence in a steppe bird community of Inner Mongolia: Insights for the conservation of the endangered Jankowski’s Bunting. Diversity and Distributions, (December 2019), 1–10.\\

Hoegh, A. \& Roberts, D. W. (2020). Evaluating and presenting uncertainty in model-based unconstrained ordination. Ecology and Evolution, 10(1), 59–69.\\

Hui, F. K. C. \textit{et al.} (2015). Model-based approaches to unconstrained ordination. Methods in Ecology and Evolution, 6(4), 399–411.\\

Hui, F. K. C. \textit{et al.} (2016a). Variational Approximations for Generalized Linear Latent Variable Models. Journal of Computational and Graphical Statistics, 26(1), 35–43.\\ 

Hui, F. K. C. (2016b). boral – Bayesian Ordination and Regression Analysis of Multivariate Abundance Data in r. Methods in Ecology and Evolution, 7(6), 744–750.\\

Hui, F. K. C. (2017). Model-based simultaneous clustering and ordination of multivariate abundance data in ecology. Computational Statistics and Data Analysis, 105, 1–10. \\

Kristensen \textit{et al.} (2016). tmb: Automatic differentiation and laplace approximation. Journal of Statistical Software, 70, 1–21. \\

Letten, A. D. \textit{et al.} (2015). Fine-scale hydrological niche differentiation through the lens of multi-species co-occurrence models. Journal of Ecology, 103(5). \\ 

Niku, J. \textit{et al.} (2017). Generalized Linear Latent Variable Models for Multivariate Count and Biomass Data in Ecology. Journal of Agricultural, Biological, and Environmental Statistics, 1–25. \\

Niku, J. \textit{et al.} (2019a). Efficient estimation of generalized linear latent vari‐ able models. PLoS ONE, 14(5), 1–20.\\

Niku, J. \textit{et al.} (2019b). gllvm – Fast analysis of multivariate abundance data with generalized linear latent variable models in R. Methods in Ecology and Evolution, 1(1), 1–2. \\

Niku, J. (2019c). On modeling multivariate abundance data with generalized linear latent variable models.

Ovaskainen, O. \textit{et al.} (2016). Using latent variable models to identify large networks of species-to-species associations at different spatial scales. Methods in Ecology and Evolution, 7(5), 549–555. \\

van der Veen, B. \textit{et al.} (2020). Model-based ordination for species with unequal niche widths. BioRxiv.\\

Walker, S. C. \& Jackson, D. A. (2011). Random-effects ordination: Describing and predicting multivariate correlations and co-occurrences. Ecological Monographs, 81(4), 635–663.\\ 

Warton, D. I. \textit{et al.} (2015). So Many Variables: Joint Modeling in Community Ecology. Trends in Ecology \& Evolution, 30(12), 766–779. 

\section{HMSC}

Baselga, A. \& Araújo, M. B. (2010). Do community-level models describe community variation effectively? Journal of Biogeography, 37(10), 1842–1850.\\

Bhattacharya, A. \& Dunson, D. B. (2011). Sparse Bayesian infinite factor models. Biometrika, 291-306.\\

Elith, J. \& Leathwick, J. R. (2009). Species Distribution Models: Ecological Explanation and Prediction Across Space and Time. Annual Review of Ecology, Evolution, and Systematics, 40(1), 677–697.\\

Felsenstein, J. (1985). Phylogenies and the comparative method. The American Naturalist, 125(1), 1-15.\\

Guisan, A. \textit{et al.}  (2017). Habitat suitability and distribution models: with applications in R. Cambridge University Press.\\

Hijmans, R. J. \& Graham, C. H. (2006). The ability of climate envelope models to predict the effect of climate change on species distributions. Global change biology, 12(12), 2272-2281.\\

Ives, A. R. \& Helmus, M. R. (2011). Generalized linear mixed models for phylogenetic analyses of community structure. Ecological Monographs, 81(3), 511–525. \\

Leathwick, J.R. \& Austin, M.P. (2001) Competitive interactions between tree species in New Zealand’s old-growth indigenous forests. Ecology, 82,2560– 2573.\\

Maguire, K. C. \textit{et al.} (2016). Controlled comparison of species- and community-level models across novel climates and communities. Philosophical Transactions of the Royal Society B, 283.\\

Norberg, A. \textit{et al.} (2019). A comprehensive evaluation of predictive performance of 33 species distribution models at species and community levels. Ecological Monographs, 89(3). \\

Opedal, Ø. H. \& Hegland, S. J. (2019). Using hierarchical joint models to study reproductive interactions in plant communities. Journal of Ecology, 2009. \\

Ovaskainen, O. \textit{et al.} (2017a). How to make more out of community data? A conceptual framework and its implementation as models and software. Ecology Letters, 20(5), 561–576.\\

Ovaskainen, O. \& Abrego, N. (2020). Joint Species Distribution Modelling: With Applications in R. Cambridge University Press.\\

Ovaskainen, O. (2020). Slides from HMSC workshop. Available here: \\https://www.helsinki.fi/en/researchgroups/statistical-ecology/hmsc\\

Peterson, A. T. \textit{et al.} (2011). Ecological niches and geographic distributions. Princeton University Press.\\

Pollock, L. J. \textit{et al.} (2014). Understanding co-occurrence by modelling species simultaneously with a Joint Species Distribution Model (JSDM). Methods in Ecology and Evolution, 5(5), 397–406.\\ 

Saine, S. \textit{et al.} (2020). Data collected by fruit body- and DNA-based survey methods yield consistent species-to-species association networks in wood-inhabiting fungal communities. Oikos, 00, 1–11.\\

Schweiger, O. \textit{et al.} (2012) Increasing range mismatching of interacting species under global change is related to their ecological characteristics. Global Ecology and Biogeography, 21,88–99. \\

Wisz, M. S. \textit{et al.} (2013). The role of biotic interactions in shaping distributions and realised assemblages of species: Implications for species distribution modelling. Biological Reviews, 88(1), 15–30.\\

Zurell, D. \textit{et al.} (2019). Testing species assemblage predictions from stacked and joint species distribution models. Journal of Biogeography, (April 2019), 101–113.\\

Zurell, D. \textit{et al.} (2019). Testing species assemblage predictions from stacked and joint species distribution models. Journal of Biogeography, (April 2019), 101–113. \\

Zurell, D. \textit{et al.} (2020). A standard protocol for reporting species distribution models. Ecography, 1–17.\\

\section{Copulas}

Anderson, M. J. \textit{et al.} (2019). A pathway for multivariate analysis of ecological communities using copulas. Ecology and Evolution, (September 2018), 1–19.\\

Genest, C. \& Favre, A.C. (2008). Everything you always wanted to know about copula modeling but were afraid to ask. Journal of Hydrologic Engineering, 13(10), 995–996.\\

Ghosh, S. \textit{et al.} (2020a). Copulas and their potential for ecology. Advances in Ecological Research, 62, 409–468. \\

Ghosh, S. \textit{et al.} (2020b). Tail associations in ecological variables and their impact on extinction risk. Ecosphere 11(5):e03132. \\

Ghosh, S. \textit{et al.} (2020c). A new approach to interspecific synchrony in population ecology using tail association. Ecology and Evolution, 00, 1–13. \\

Humphreys, R. K. \textit{et al.} (2018). Underestimation of Pearson’s product moment correlation statistic. Oecologia\\

Popovic, G. C. \textit{et al.} (2018). A general algorithm for covariance modeling of discrete data. Journal of Multivariate Analysis, 165, 86–100.\\

Popovic, G. C. \textit{et al.} (2019). Untangling direct species associations from indirect mediator species effects with graphical models. Methods in Ecology and Evolution, 2019(June), 1571–1583.\\

Sklar, A. (1959). Fonctions de répartition à n dimensions et leurs marges. Publications De L’institut De Statistique De L’université De Paris, 8, 229–231\\ 

Terui, A. \textit{et al.} (2019). Metapopulation stability in branching river networks. Pnas, 116(3), 1067–1068.\\

\section{Not covered - but interesting}

Clark, N. J. \textit{et al.} (2018). Unravelling changing interspecific interactions across environmental gradients using Markov random fields. Ecology, 99(6), 1277–1283\\

Hawinkel, S. et al. (2019). A unified framework for unconstrained and constrained ordination of microbiome read count data. PLoS ONE, 14(2), 1–20. https://doi.org/10.1101/429340

Roberts, D. W. (2008). Statistical Analysis of Multidimensional Fuzzy Set Ordinations. Ecology, 89(5), 1246–1260.\\



\end{document}